\section{Introduction}

%% TODO: Talk about latest in the field of handwriting recognition
%% TODO: Gather some statistics about the Indexing program of the LDS Church
%% TODO: Add and refine a humorous introduction
%% TODO: Talk about why this problem is interesting


The Church of Jesus Christ of Latter-Day Saints does extensive work with digitizing census records into a digital format.  This is done manually and all over the world.
In 2015, 330 million records were manually indexed through the indexing program of the Church of Jesus Christ of Latter-Day Saints (LDS Church) \cite{web:indexingStats}.
Today alone, 88,000 records were indexed by individuals all around the world \cite{web:indexingStats}.
There are millions of records today waiting to be indexed so that members and non-members alike can connect the dots in compiling their family history.
There is so much information stored in handwritten form that currently we rely on people to descipher what is written.

We decided to try our hand at using machine learning to automate indexing of historical documents.  In the general case, handwriting recognition is an extremely difficult as there can be wild variations in handwriting style, as well as corrections and mistakes.  To limit the scope of our project, we limited the task to recognizing a handwritten M or F in the gender field of an old census record.

Using data we obtained from the LDS Church we processed, cleaned, ran various learning algorithms.  This included classifiers we learned from class and we also explored using Deep Learning.  Of these methods, Deep Learning proved the most effective, however pretty strong results came from linear classifiers.  We learned that success in Machine Learning has much to do with activities other than machine learning algorithms, such as gathering, cleaning, and preprocessing data.

